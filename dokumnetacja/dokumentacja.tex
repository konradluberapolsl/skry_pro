\documentclass[12pt,a4paper]{article}
\usepackage[polish]{babel}
\usepackage[T1]{fontenc}
\usepackage[utf8x]{inputenc}
\usepackage{hyperref}
\usepackage{url}
\usepackage{graphicx}

\addtolength{\hoffset}{-1.5cm}
\addtolength{\marginparwidth}{-1.5cm}
\addtolength{\textwidth}{3cm}
\addtolength{\voffset}{-1cm}
\addtolength{\textheight}{2.5cm}
\setlength{\topmargin}{0cm}
\setlength{\headheight}{0cm}

\begin{document}
	
	\title{Dokumentacja projektu\\ Języki Skryptowe}
	\author{Konrad Lubera, gr 1A}
	\date{\today}
	
	\maketitle
	\newpage
	\section*{Część I}
	\subsection*{Opis programu}
	Ogólne informacje o przydatności programu i cel (np.: polecenie). 
	\subsection*{Instrukcja obsługi}
	Opis jak właczyć i obsługiwać program. Można dodać zrzuty ekranu (dokładniej, wycinki).
\newpage
	\section*{Część II}
	\subsection*{Część techniczna}
	Schemat działania i szczegółowy opis komponentów, mechanizmów (bez kodu).
	\subsection*{Opis działania} 
	Tutaj uwzględniamy część matematyczną. Wszelki opis działania programu, komponentów (bez kodu)
	\subsection*{Implementacja}
	Tutaj opisujemy wybrane działanie KODU programu -- z uwzględnieniem PSEUDOKODU (utworzonego przy użyciu \LaTeX, a nie obrazka).
	\begin{verbatim}
	proszę zwrócić uwagę na wychodzenie poza obszar kartki.
	\end{verbatim}
	\newpage
	\section*{Pełen kod programu}

	Wklejamy pełen kod z podziałem na pliki, np.:
	
	\begin{itemize}
	\item PLIK I
	\begin{verbatim}
	kod pliku I
	\end{verbatim}
	\item PLIK II
	\begin{verbatim}
	kod pliku II
	\end{verbatim}
	\end{itemize}
\end{document}
